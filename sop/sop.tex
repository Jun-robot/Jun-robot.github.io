\documentclass[12pt]{article}

% ---- Page layout ----
\usepackage[letterpaper,margin=1in]{geometry}

% ---- Font ----
\usepackage{times} % Times New Roman 相当

% ---- Line spacing ----
\usepackage{setspace}
\setstretch{1.2}

% ---- No paragraph indent, small spacing instead ----
\setlength{\parindent}{0pt}
\setlength{\parskip}{0.6em}

% ---- Disable page numbers ----
\pagestyle{empty}

\begin{document}

\begin{center}
{\Large \textbf{Statement of Purpose}}\\
\vspace{0.5em}
{\normalsize Jumpei Saito}\\
{\normalsize CMU Robotics Institute Summer Scholars (RISS) Program}
\end{center}

\vspace{1em}
% Paragraph0
I build robotic systems by integrating hardware and software into working physical machines, as demonstrated by leading a team to success at an international robotics competition. Through this experience, I became interested in research on coordinated systems, where multiple components work together to achieve a single physical goal. At RISS, I want to study such systems as real machines and develop them into research contributions.

% Paragraph1
At RoboCup, an international robotics competition, I led a team to first place by building a reliable robotic system. In a real-world setting, I helped turn an idea into a robust physical platform that could be evaluated. I contributed across hardware and mechanical integration, low-level control, and software implementation. For example, I designed a custom low-latency communication architecture using RS485 and ZigBee to achieve synchronized real-time behavior across 12 robots, comprising over 50 microcontrollers.

% Paragraph2
Through competition, I gained experience designing and implementing solutions for well-defined problems. I am now interested in research problems where the correct answer and evaluation criteria are not predefined, and where the questions themselves must be designed and validated.

% % Paragraph3
% Through my experience building individual systems, I became increasingly interested in problems that cannot be solved by a single component alone.
% In academia, I want to study systems where many components cooperate to achieve a single physical goal.

% Paragraph 4
This perspective on coordinated physical systems is reflected in several research projects at the Robotics Institute.

PuzzleBots, developed by Zeynep Temel and Katia Sycara, studies how multiple robots can work together through physical connections. In this work, robots can achieve tasks that a single robot cannot by acting as a connected group. I am interested in building software that allows robots to make autonomous decisions on how and when to cooperate, grounded in their physical capabilities.

In Solid Knitting, led by James McCann (though not listed, I have a strong interest in his lab), simple elements such as needles and stitches are used to build three-dimensional objects. I am drawn to how complex physical structures emerge from many simple actions through tight integration of software and machines. I am also interested in how such systems could be extended through improvements in machine capability and control.

I also see related ideas in projects such as Linear Delta Arrays by Zeynep Temel and Oliver Kroemer, RoboLoom by Melisa Orta Martinez, and Modular EigenBots by Howie Choset, where coordinated behavior emerges from many simple robotic or mechanical units.
These projects motivate my interest in studying coordination not only as an algorithm, but as a system-level problem shaped by hardware, control, and physical interaction.

At RISS, I want to build and study such systems to understand how design choices at the hardware and control levels affect system performance. Through this process, I aim to turn hands-on implementation into research insights.

% Paragraph5
Beyond technical implementation, I am also committed to contributing to the developer community, as demonstrated by founding a student robotics expo in Japan to connect young developers with experts.

I am especially interested in the RISS program because it focuses on short-term, project-based research with clear outcomes. I want to contribute as a hands-on researcher who can quickly turn ideas into working physical systems, especially in projects that require close integration of hardware, low-level control, and software. Based on my experience building reliable robotic systems with many components, I am ready to take responsibility for system-level implementation while working closely with faculty mentors. 

Through RISS, I hope to learn how coordinated robotic systems can be developed into research contributions suitable for publication within the program's timeframe, as preparation for pursuing a PhD.
\end{document}
