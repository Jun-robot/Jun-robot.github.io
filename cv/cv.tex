% !TEX program = lualatex
\documentclass[11pt,a4paper]{article}

\usepackage{cv} % <- cv.sty
\usepackage{fontspec} % LuaLaTeX
% \setmainfont{TeX Gyre Termes} % Times系(TeX Live標準)
% \setsansfont{TeX Gyre Heros}

\setmainfont{Helvetica Neue}
\setsansfont{Helvetica Neue}
\renewcommand{\familydefault}{\sfdefault}
% \linespread{1.15}
% \setlength{\parskip}{3pt}

\usepackage[none]{hyphenat}


\begin{document}

% ===== Header =====
\cvheader
  {Jumpei Saito} % Name
  {Keio University | 2nd-year Undergraduate | Faculty of Environment and Information Studies} % Tagline
  {
    Kanagawa, Japan \quad|\quad
    Email: jumpei.saito@keio.jp \quad|\quad
    GitHub: Jun-robot \quad|\quad
    Website: https://jun-robot.github.io
  }

\vspace{-5mm}

% ===== Sections =====
% ---------- Research Interests ----------
\sectiontitle{Research Interests}
Digital Fabrication, Multi-Robot Systems, Embedded \& Wireless Systems, Human-Computer Interaction

% ---------- Education ----------
\sectiontitle{Education}
\cventry
  {Bachelor (Expected), Faculty of Environment and Information Studies (SFC)}
  {Keio University}
  {Apr 2024 -- Mar 2028 (expected)}
  {}
  {
  Major: Computer Science { } GPA: 3.68 / 4.00
  }

% ---------- Research Experience ----------
\sectiontitle{Research Experience}
\cventry
  {Undergraduate Researcher}
  {Auto-ID Laboratory, Keio University — Advisor: Jin Mitsugi}
  {Oct 2025 -- Present}
  {Research on backscatter communication and signal processing.}
  {\cvbullets{
    \item Designed and implemented QPSK signal processing for backscatter communication.
    \item Implemented modulation/demodulation and evaluated communication performance using MATLAB.
  }}

\cventry
  {Technical Staff}
  {Programmable Products Lab, Keio University — Advisor: Koya Narumi}
  {Mar 2025 -- Present}
  {Technical support and system implementation for digital fabrication research.}
  {\cvbullets{
  \item Implemented hardware, electronics, and control software for a roll-to-roll heat-sealing fabrication machine.
  \item Developed a custom G-code-driven control software for fabrication workflows.
  \item Built research-grade prototypes and data-collection apparatus for digital fabrication studies.
  }}

\cventry
  {Research Mentee}
  {Experts in Information Science Program, National Institute of Informatics (NII) — Advisor: Koya Narumi}
  {Apr 2023 -- Mar 2024}
  {Research on interactive display systems using autonomous mobile robots.}
  {\cvbullets{
    % \item Selected as one of 40 high school students nationwide for a highly competitive informatics research program at NII.
    \item Investigated methods for presenting images larger than physical screens through autonomous moving displays.
    \item Presented research outcomes through poster and oral sessions; awarded Best Presentation / Poster Award.
  }}

% ---------- publications ----------
\sectiontitle{Publications and Research Outputs (in Japanese)}
{\small English titles are provided for reference.}
\cvbullets{
  \item[1] \textbf{Jumpei Saito}, and Koya Narumi. 2024. 
  An Interactive Method for Displaying Images Larger than a Screen Using a Mobile Actuated Display. 
  Poster presentation at the 86th National Convention of IPSJ.
  \item[2] \textbf{Jumpei Saito}, Ryuki Tsuji, and Tomohiko Iida. 2025. 
  Challenge and Future of Interface Development for Representing Virtual Worlds, following the Overall Championship at RoboCup 2024 OnStage League.
  Invited talk at the WIDE Camp, March 2025.
  \item[3] Rin Ishiguro, \textbf{Jumpei Saito}, Takumi Yamamoto, Mayuka Kuwana, and Koya Narumi. 2026. 
  Implementing Deployable and Freeform Balloon Structures with a Single-Stroke Pouch Motor. 
  The 216th IPSJ SIGHCI Technical Report, vol. 2026-HCI-216, no. 29, 7 pages. \href{https://ipsj.ixsq.nii.ac.jp/record/2006471/files/IPSJ-HCI26216029.pdf}{[Link]}
  \item[4] Hiroto Horie, Daniel Campos Zamora, Liang He, \textbf{Jumpei Saito}, and Koya Narumi. 2026. 
  Implementing a Mobile 3D Printer that can Localize a Printing Position via Natural Language Instruction. 
  The 88th National Convention of IPSJ, 2 pages, to appear in 2026.
  \item[5] Kaito Kikuchi, \textbf{Jumpei Saito}, Takeo Igarashi, and Koya Narumi. 2026. 
  A Method to Synthesize Worn-Out Fabric Texture with a Physical Image Dataset. 
  The 88th National Convention of IPSJ, 2 pages, to appear in 2026.
}

% ---------- Projects ----------
\sectiontitle{Projects}
\cventry
  {RoboCup OnStage Team ``Tomoshibi Technology''}
  {Team Leader \& Founder}
  {Apr 2023 -- Present}
  {Development of interactive multi-robot systems integrating mobility, actuation, and visual expression.}
  {\cvbullets{
    \item Founded and led a robotics team from scratch, mentoring members with no prior robotics experience.
    \item Designed and implemented interactive systems, including mobile displays and illuminated actuated mechanisms.
    % \item Led full-stack system integration spanning mechanical design (CAD), digital fabrication, circuit design (KiCad), embedded software, and FPGA-based control.
    \item Built and deployed a coordinated system of up to 15 robots; won championship titles at national and international RoboCup OnStage competitions.
    [ \href{https://www.youtube.com/watch?v=E2yGFrZC9-w&list=PLl3hQ24FuOuhNpgphr6uVN_bDWF8NaO8i&index=2}{Technical Demonstration Link} | \href{https://www.youtube.com/watch?v=eePvvMbTeU4&list=PLl3hQ24FuOuhNpgphr6uVN_bDWF8NaO8i&index=3}{Performance Video Link} | \href{https://drive.google.com/file/d/19TOM5o9O2XdSBc00Fr5O3k87qQRkatmN/preview}{Poster Link} ]
  }}

% \cventry
%   {Swarm Robots as a Medium for Ecosystem Dynamics}
%   {Exploratory Project}
%   {Sep 2025 -- Present}
%   {Exploring swarm robots as an interactive medium for representing ecosystem dynamics.}
%   {\cvbullets{
%     \item Implemented a swarm robot coordination system for ~20 Sony toio robots with collision avoidance.
%     \item Demonstrated the system in collaboration with researchers in audio signal processing and biological systems.
%   }}

% ---------- Research Presentations ----------
% \sectiontitle{Research Presentations}
% {\small
% \cvbullets{
%   \item \textit{Spring 2025} --- Invited Oral Presentation, WIDE Camp: ``RoboCup 2024 OnStage League Overall Championship: Challenges and Future Directions in Developing Interfaces for Expressing Virtual Worlds.''
%   \item \textit{2024} --- Poster Presentation, IPSJ 86th National Convention: ``An Interactive Display Method for Presenting Images Larger Than Physical Screens Using Autonomous Mobile Displays.''
% }}


\sectiontitle{Technical Skills}

\cvgroup{Software}{}
\cvbullets{
  \item C/C++, Python, Verilog, MATLAB, Linux, Git
  \item Embedded software (motor control, timing-critical control, inter-MCU communication)
  \item HAL-based MCU and FPGA programming for communication (STM32, ESP32, Gowin Tang series)
}

\cvgroup{Hardware}{}
\cvbullets{
  \item 3D printing (Ender-3, Guider 2S, Bambu Lab printers), CNC machining (KitMill CL200), Laser cutting
  \item Mechanical design using Autodesk Fusion, and Schematic and PCB design using KiCad
}




% Awards
\sectiontitle{Awards}
\cvbullets{
  \cvaward{2024}{RoboCup Eindhoven 2024 OnStage League}{Individual Team 1st Place}{International performance-robotics competition; won 1st place among 24 invited teams from around the world.}
  \cvaward{2024}{RoboCupJunior Japan Open 2024 OnStage League}{Champion \& Presentation Award}{Won the championship after six consecutive years of participation; qualified for the world championship.}
  \cvaward{2024}{Experts in Information Science Program (NII)}{Best Poster Presentation Award}{Received the program’s top award among 40 selected students in NII’s research mentorship program.}
  % \cvaward{2024}{86th National Convention of the Information Processing Society of Japan (IPSJ)}{Best Poster Presentation Award}{Awarded for outstanding poster presentation on interactive display methods using autonomous mobile robots.}

  \cvaward{2025}{Keio University Shonan Fujisawa Campus (SFC)}{SFC STUDENT AWARD}{Faculty-wide award; first-year undergraduate recipient in 9-years.}
}


% Fellowships \& Grants
\sectiontitle{Fellowships \& Grants}
\cvbullets{
  \cvaward{2022 -- Present}{Masason Foundation Fellow}{Full Scholarship}{One of 30 global fellows selected by Masayoshi Son (CEO of SoftBank Group) received cumulative funding exceeding JPY 10,000,000 for education and independent research.}
  \cvaward{2023}{MITOU Junior Program}{Selected Creator}{One of 20 under-18 participants nationwide; awarded a JPY 500,000 scholarship to develop a custom 3D-printed motor with mentorship. [ \href{https://jr.mitou.org/english/projects/2023/3d_printed_brushless_motor}{Project Page} ]}
  % \cvaward{2024}{External Project Funding (Crowdfunding)}{}{Secured JPY 3,055,000 from 168 supporters to support international competition participation (RoboCup Eindhoven 2024).}
  % \href{https://camp-fire.jp/projects/view/675041}{https://camp-fire.jp/projects/view/675041}
}

% Leadership & Outreach
\sectiontitle{Leadership \& Outreach}
\cvbullets{
  \item Successfully crowdfunded JPY 3,055,000 from 168 supporters to cover travel costs for the RoboCup Eindhoven 2024 competition. [ \href{https://camp-fire.jp/projects/762605/view}{Crowdfunding Project Page (JP)} ]
  \item Founded and organized a student robotics and hardware expo in Japan (Kanto region), connecting student developers and companies (scheduled for Feb 2026; JPY 1,000,000 funding; 50+ participants). [ \href{https://robot.exposition.tokyo/}{Event Website (JP)} ]
}

\end{document}
