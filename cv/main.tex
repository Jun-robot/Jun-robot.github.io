% !TEX program = lualatex
\documentclass[11pt,a4paper]{article}

\usepackage{cv} % <- cv.sty
\usepackage{fontspec} % LuaLaTeX
% \setmainfont{TeX Gyre Termes} % Times系(TeX Live標準)
% \setsansfont{TeX Gyre Heros}

\setmainfont{Helvetica Neue}
\setsansfont{Helvetica Neue}
\renewcommand{\familydefault}{\sfdefault}
% \linespread{1.15}
% \setlength{\parskip}{3pt}

\usepackage[none]{hyphenat}


\begin{document}

% ===== Header =====
\cvheader
  {Jumpei Saito} % Name
  {Keio University | Faculty of Environment and Information Studies} % Tagline
  {
    Kanagawa, Japan \quad|\quad
    mail: jumpei.saito@keio.jp \quad|\quad
    GitHub: Jun-robot \quad|\quad
    Website: jun-robot.github.io
  }

\vspace{2mm}

% ===== Sections =====
% ---------- Research Interests ----------
\sectiontitle{Research Interests}
\cvbullets{
  \item Robotics, Embedded Systems, Human-Computer Interaction, Wireless Communications, Interactive Robotic Systems
}

% ---------- Education ----------
\sectiontitle{Education}
\cventry
  {Bachelor (Expected), Faculty of Environment and Information Studies (SFC)}
  {Keio University}
  {Apr 2024 -- Mar 2028 (expected)}
  {}
  {\cvbullets{
    \item Focus: Wireless Communications, HCI, Robotics.
    \item GPA: 3.69 / 4.00
  }}

% ---------- Projects ----------
\sectiontitle{Projects}
\cventry
  {RoboCup OnStage Team ``Tomoshibi Technology''}
  {Team Leader \& Founder}
  {Apr 2023 -- Present}
  {Development of interactive multi-robot systems integrating mobility, actuation, and visual expression.}
  {\cvbullets{
    \item Founded and led a robotics development team, Tomoshibi Technology, mentoring members with no prior robotics experience.
    \item Designed and implemented interactive robotic systems, including mobile displays and illuminated robotic arms.
    \item Integrated full-stack development including mechanical design (CAD), 3D fabrication, circuit design (KiCad), embedded software, and FPGA-based control.
    \item Built a coordinated system of up to 15 robots and achieved championship titles at national and international competitions (RoboCup Junior OnStage).
  }}

\cventry
  {Swarm Robots as a Medium for Ecosystem Dynamics}
  {Exploratory Project}
  {Sep 2025 -- Present}
  {Research on swarm robot coordination and interdisciplinary interactive expression.}
  {\cvbullets{
    \item Designed and implemented a swarm robot coordination system for approximately 20 Sony toio robots with collision avoidance.
    \item Collaborated with researchers in audio signal processing and biological systems to explore mappings between swarm robot behaviors and ecosystem dynamics.
  }}

% ---------- Research Experience ----------
\sectiontitle{Research Experience}
\cventry
  {Undergraduate Researcher}
  {Auto-ID Laboratory, Keio University — Advisor: Jin Mitsugi}
  {Oct 2025 -- Present}
  {Research on low-power wireless communication and signal processing.}
  {\cvbullets{
    \item Designed and implemented QPSK baseband signal processing for backscatter communication.
    \item Implemented modulation/demodulation and evaluated communication performance using MATLAB.
  }}

\cventry
  {Technical Staff}
  {Narumi Laboratory, Keio University — Advisor: Koya Narumi}
  {Mar 2025 -- Present}
  {Technical support and system implementation for digital fabrication research.}
  {\cvbullets{
  \item Implemented hardware, electronics, and control software for a pouch-actuator-based fabrication machine.
  \item Developed G-code-driven control software for fabrication workflows.
  \item Built custom experimental apparatuses used for digital fabrication research and data collection.
  }}

\cventry
  {Research Mentee}
  {Experts in Information Science Program, National Institute of Informatics (NII) — Advisor: Koya Narumi}
  {Apr 2023 -- Mar 2024}
  {Research on interactive display systems using autonomous mobile robots.}
  {\cvbullets{
    \item Selected as one of 40 high school students nationwide for a highly competitive informatics research program at NII.
    \item Investigated methods for presenting images larger than physical screens through autonomous moving displays.
    \item Presented research outcomes through poster and oral sessions; awarded Best Presentation / Poster Award.
  }}

% ---------- Publications / Posters ----------
\sectiontitle{Publications / Posters}
\cvlistentry
  {Invited Oral Presentation — WIDE Camp}
  {Spring 2025}
  {``RoboCup 2024 OnStage League Overall Championship: Challenges and Future Directions in Developing Interfaces for Expressing Virtual Worlds.''}

\cvlistentry
  {Poster Presentation — IPSJ 86th National Convention}
  {2024}
  {``An Interactive Display Method for Presenting Images Larger Than Physical Screens Using Autonomous Mobile Displays.''}


% Technical Skills
\sectiontitle{Technical Skills}
\cvgroup{Programming}{}
\cvbullets{
  \item C/C++, Python, (others)
  \item ROS2, embedded HAL/RTOS (if used), Linux, Git
}

\cvgroup{Hardware}{}
\cvbullets{
  \item STM32, ESP32, FPGA (if applicable), KiCad, sensors/actuators
}



% Awards
\sectiontitle{Awards}
\cvgroup{International \& National Awards}{2019 -- 2024}
\cvbullets{
  \item RoboCup Eindhoven 2024 OnStage League --- 1st Place (Individual Team).
  \item RoboCupJunior 2024 Japan Open OnStage --- Champion \& Presentation Award.
  \item Best Poster / Presentation Award, IPSJ National Convention.
  \item Best Award, Experts in Information Science Program.
  \item SFC Award, Keio University.
}


% Fellowships
\sectiontitle{Fellowships}
\cvgroup{Selected Fellowships}{2022 -- Present}
\cvbullets{
  \item Masason Foundation Fellow.
  \item Experts in Information Science Program.
  \item Mitou Junior Program.
  \item SecHack365.
  \item Makers University U18.
}



% Activities
\sectiontitle{Activities}
\cvgroup{Leadership \& Community Activities}{2023 -- Present}
\cvbullets{
  \item Leader, Tomoshibi Technology.
  \item Organizer and technical contributor, Student Robot Expo.
}


\end{document}
