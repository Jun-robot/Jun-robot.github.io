% !TEX program = lualatex
\documentclass[11pt,a4paper]{article}

\usepackage{cv} % <- cv.sty
\usepackage{fontspec} % LuaLaTeX
% \setmainfont{TeX Gyre Termes} % Times系(TeX Live標準)
% \setsansfont{TeX Gyre Heros}

\setmainfont{Helvetica Neue}
\setsansfont{Helvetica Neue}
\renewcommand{\familydefault}{\sfdefault}
% \linespread{1.15}
% \setlength{\parskip}{3pt}

\usepackage[none]{hyphenat}


\begin{document}

% ===== Header =====
\cvheader
  {Jumpei Saito} % Name
  {Keio University | 2nd-year Undergraduate | Faculty of Environment and Information Studies} % Tagline
  {
    Kanagawa, Japan \quad|\quad
    mail: jumpei.saito@keio.jp \quad|\quad
    GitHub: Jun-robot \quad|\quad
    Website: jun-robot.github.io
  }

\vspace{-2mm}

% ===== Sections =====
% ---------- Research Interests ----------
\sectiontitle{Research Interests}
Robotics, Embedded Systems, Human-Computer Interaction, Wireless Communications, Interactive Robotic Systems

% ---------- Education ----------
\sectiontitle{Education}
\cventry
  {Bachelor (Expected), Faculty of Environment and Information Studies (SFC)}
  {Keio University}
  {Apr 2024 -- Mar 2028 (expected)}
  {}
  {\cvbullets{
    \item Focus: Wireless Communications, HCI, Robotics.
    \item GPA: 3.68 / 4.00
  }}

% ---------- Research Experience ----------
\sectiontitle{Research Experience}
\cventry
  {Undergraduate Researcher}
  {Auto-ID Laboratory, Keio University — Advisor: Jin Mitsugi}
  {Oct 2025 -- Present}
  {Research on backscatter communication and signal processing.}
  {\cvbullets{
    \item Designed and implemented QPSK signal processing for backscatter communication.
    \item Implemented modulation/demodulation and evaluated communication performance using MATLAB.
  }}

\cventry
  {Technical Staff}
  {Programmable Products Lab, Keio University — Advisor: Koya Narumi}
  {Mar 2025 -- Present}
  {Technical support and system implementation for digital fabrication research.}
  {\cvbullets{
  \item Implemented hardware, electronics, and control software for a pouch-actuator-based fabrication machine.
  \item Developed G-code-driven control software for fabrication workflows.
  \item Built custom experimental apparatuses used for digital fabrication research and data collection.
  }}

\cventry
  {Research Mentee}
  {Experts in Information Science Program, National Institute of Informatics (NII) — Advisor: Koya Narumi}
  {Apr 2023 -- Mar 2024}
  {Research on interactive display systems using autonomous mobile robots.}
  {\cvbullets{
    \item Selected as one of 40 high school students nationwide for a highly competitive informatics research program at NII.
    \item Investigated methods for presenting images larger than physical screens through autonomous moving displays.
    \item Presented research outcomes through poster and oral sessions; awarded Best Presentation / Poster Award.
  }}

  % ---------- Projects ----------
\sectiontitle{Projects}
\cventry
  {RoboCup OnStage Team ``Tomoshibi Technology''}
  {Team Leader \& Founder}
  {Apr 2023 -- Present}
  {Development of interactive multi-robot systems integrating mobility, actuation, and visual expression.}
  {\cvbullets{
    \item Founded and led a robotics development team, Tomoshibi Technology, mentoring members with no prior robotics experience.
    \item Designed and implemented interactive robotic systems, including mobile displays and illuminated robotic arms.
    \item Integrated full-stack development including mechanical design (CAD), 3D fabrication, circuit design (KiCad), embedded software, and FPGA-based control.
    \item Built a coordinated system of up to 15 robots and achieved championship titles at national and international competitions (RoboCup Junior OnStage).
  }}

\cventry
  {Swarm Robots as a Medium for Ecosystem Dynamics}
  {Exploratory Project}
  {Sep 2025 -- Present}
  {Development on swarm robot coordination and interdisciplinary interactive expression.}
  {\cvbullets{
    \item Designed and implemented a swarm robot coordination system for approximately 20 Sony toio robots with collision avoidance.
    \item Collaborated with researchers in audio signal processing and biological systems to explore mappings between swarm robot behaviors and ecosystem dynamics.
  }}

% ---------- Research Presentations ----------
% \sectiontitle{Research Presentations}
% {\small
% \cvbullets{
%   \item \textit{Spring 2025} --- Invited Oral Presentation, WIDE Camp: ``RoboCup 2024 OnStage League Overall Championship: Challenges and Future Directions in Developing Interfaces for Expressing Virtual Worlds.''
%   \item \textit{2024} --- Poster Presentation, IPSJ 86th National Convention: ``An Interactive Display Method for Presenting Images Larger Than Physical Screens Using Autonomous Mobile Displays.''
% }}


\sectiontitle{Technical Skills}

\cvgroup{Software}{}
\cvbullets{
  \item C/C++, Python, Verilog, MATLAB, Linux, Git
  \item Embedded software (motor control, robot internal networking, etc...)
  \item HAL-based MCU programming (STM32, ESP32, ATmega)
  \item FPGA-based communication and rendering pipelines (Tang Nano, Tang Primer)
}

\cvgroup{Hardware}{}
\cvbullets{
  \item 3D printing (Ender-3, Guider 2S, Bambu Lab)
  \item CNC machining (KitMill CL200), laser cutting
  \item Mechanical design using Autodesk Fusion
  \item Schematic and PCB design using KiCad
}




% Awards
\sectiontitle{Awards}
\cvbullets{
  \cvaward{2024}{RoboCup Eindhoven 2024 OnStage League}{Individual Team 1st Place}{International performance-robotics competition; won 1st place among 24 invited teams from around the world.}
  \cvaward{2024}{RoboCupJunior Japan Open 2024 OnStage League}{Champion \& Presentation Award}{Won the championship after a 6-year journey of continual challenge; earned an invitation to the world championship.}
  \cvaward{2024}{Experts in Information Science Program (NII)}{Best Poster Presentation Award}{Received the program’s top award among 40 selected high school students in NII’s research mentorship program.}
  % \cvaward{2024}{86th National Convention of the Information Processing Society of Japan (IPSJ)}{Best Poster Presentation Award}{Awarded for outstanding poster presentation on interactive display methods using autonomous mobile robots.}

  \cvaward{2025}{Keio University Shonan Fujisawa Campus (SFC)}{SFC STUDENT AWARD}{Faculty-wide award; first-year undergraduate recipient in 9-years.}
}


% Fellowships \& Grants
\sectiontitle{Fellowships \& Grants}
\cvbullets{
  \cvaward{2022 -- Present}{Masason Foundation Fellow}{Full Scholarship}{One of 30 global fellows selected by Masayoshi Son (SoftBank Group); received funding exceeding JPY 10,000,000 for education and independent research.}
  \cvaward{2023}{MITOU Junior Program}{Selected Creator}{One of 20 under-18 participants nationwide; awarded a JPY 500,000 scholarship to develop a custom 3D-printed motor with mentorship.}
}

% Leadership & Outreach
\sectiontitle{Leadership \& Outreach}
\cvbullets{
  \item Successfully crowdfunded JPY 3,055,000 from 168 supporters to cover travel costs for the RoboCup Eindhoven 2024 competition.
  \item Founded and organized a student robotics and hardware expo with confirmed funding of JPY 1,000,000 and 50+ expected participants (February 2026).
}

\end{document}
